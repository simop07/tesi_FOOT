\documentclass[12pt,a4paper]{report}
\usepackage[italian]{babel}
\usepackage{newlfont}
\usepackage{color}
\textwidth=450pt\oddsidemargin=0pt

% INIZIO HEADER INSERITI DA ME, DA CUI HO TOLTO GLI HEADER CHE SI RIPETONO

%\documentclass[12pt,a4paper]{article}
%\usepackage[utf8]{inputenc}
\usepackage{titling}
\usepackage[backend=bibtex,
			style=numeric,
			sorting=none
			]{biblatex}
\addbibresource{bibliography.bib} %Import the bibliography file
%\setlength{\droptitle}{-2cm}% change title position
%\usepackage{amsmath}
%\usepackage{amsfonts}
%\usepackage{amssymb}
%\usepackage[T1]{fontenc}
%\usepackage{multirow}
%\usepackage{array}
%\newcolumntype{P}[1]{>{\centering\arraybackslash}p{#1}}
%\newcolumntype{M}[1]{>{\centering\arraybackslash}m{#1}}
%\usepackage{graphicx}
%\usepackage{siunitx}
\usepackage{hyperref}
%\usepackage{float}
%\usepackage[margin=2cm]{geometry}
%\usepackage{listings}
%\usepackage{circuitikz}
%\usepackage{subcaption}
%\usepackage{tabularx}
\hypersetup{
	colorlinks,
	citecolor=black,
	filecolor=black,
	linkcolor=black,
	urlcolor=black
}
%\renewcommand{\lstlistingname}{Code}
%\renewcommand{\lstlistlistingname}{List of Code}
%\lstdefinestyle{chstyle}{
	%	backgroundcolor=\color{gray!12},
	%	basicstyle=\ttfamily\small,
	%	commentstyle=\color{green!60!black},
	%	keywordstyle=\color{magenta},
	%	stringstyle=\color{blue!50!red},
	%	showstringspaces=false,
	%	%captionpos=b,
	%	numbers=left,
	%	numberstyle=\footnotesize\color{gray},
	%	numbersep=10pt,
	%	%stepnumber=2,
	%	tabsize=2,
	%	frame=L,
	%	framerule=1pt,
	%	rulecolor=\color{red},
	%	breaklines=true,
	%	inputpath=code
	%}
%\renewmenumacro{\directory}{pathswithfolder} % default: path
%\renewmenumacro{\keys}{shadowedroundedkeys} % default: roundedkeys
%\setlength{\arrayrulewidth}{1.0pt}
%\renewcommand{\arraystretch}{1}

%\usepackage{fontspec}
%\setmainfont{Calibri}

% è già di default interlinea singola
%\usepackage{setspace}
%\singlespacing

%\usepackage[font=it]{caption}
%\usepackage{indentfirst}

% FINE HEADER INSERITI DA ME

\begin{document}
	\begin{titlepage}
		%
		%
		% UNA VOLTA FATTE LE DOVUTE MODIFICHE SOSTITUIRE "RED" CON "BLACK" NEI COMANDI \textcolor
		%
		%
		\begin{center}
			{{\Large{\textsc{Alma Mater Studiorum $\cdot$ Universit\`a di Bologna}}}} 
			\rule[0.1cm]{15.8cm}{0.1mm}
			\rule[0.5cm]{15.8cm}{0.6mm}
			\\\vspace{3mm}
			
			{\small{\bf Scuola di Scienze \\ 
					Dipartimento di Fisica e Astronomia\\
					Corso di Laurea in Fisica}}
			
		\end{center}
		
		\vspace{23mm}
		
		\begin{center}\textcolor{red}{
				%
				% INSERIRE IL TITOLO DELLA TESI
				%
				{\LARGE{\bf TITOLO TESI}}\\
		}\end{center}
		
		\vspace{50mm} \par \noindent
		
		\begin{minipage}[t]{0.47\textwidth}
			%
			% INSERIRE IL NOME DEL RELATORE CON IL RELATIVO TITOLO DI DOTTORE O PROFESSORE
			%
			{\large{\bf Relatore: \vspace{2mm}\\\textcolor{red}{
						Prof./Dott. Nome Cognome}\\\\
					%
					% INSERIRE IL NOME DEL CORRELATORE CON IL RELATIVO TITOLO DI DOTTORE O PROFESSORE
					%
					% SE NON AVETE UN CORRELATORE CANCELLATE LE PROSSIME 3 RIGHE
					%
					\textcolor{red}{
						\bf Correlatore: (eventuale)
						\vspace{2mm}\\
						Prof./Dott. Nome Cognome\\\\}}}
		\end{minipage}
		%
		\hfill
		%
		\begin{minipage}[t]{0.47\textwidth}\raggedleft {
				{\large{\bf Presentata da:
						\vspace{2mm}\\
						Simone Pasquini}}}
		\end{minipage}
		
		\vspace{40mm}
		
		\begin{center}
			%
			% INSERIRE L'ANNO ACCADEMICO
			%
			Anno Accademico \textcolor{red}{ 201*/201*}
		\end{center}
		
	\end{titlepage}
	\newpage
	
	\chapter*{Sommario}
		Questo è l'inizio del sommario.
	\newpage
	\tableofcontents
	\newpage
	\addcontentsline{toc}{chapter}{Introduzione}
	\chapter*{Introduzione}
		Questo è l'inizio dell'introduzione.
	\newpage
	\addcontentsline{toc}{chapter}{Conclusioni}
	
	\chapter{Primo capitolo}
	
	\chapter*{Conclusioni}
		Let's cite! Einstein's journal paper \cite{einstein} and Dirac's book \cite{dirac} are physics-related items.
	\newpage	
	\printbibliography[
		heading=bibintoc,
		title={Bibliografia}
		]
		 	
\end{document}

